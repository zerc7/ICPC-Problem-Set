\documentclass[a4paper]{article}
\usepackage{lipsum}
\usepackage{amsmath,amsthm,amssymb,amsfonts}

\usepackage{tabularx}
\usepackage{graphicx}
\usepackage{geometry}
\geometry{left=2cm, right=2cm, top=2.5cm, bottom=2.5cm}
\usepackage{listings}
\usepackage{fancyhdr} % Headers and footers
\usepackage{lastpage}
\pagestyle{fancy} % All pages have headers and footers
\fancyfoot[C]{\textsf{Page \thepage \hspace{1pt} of \pageref{LastPage}}}
\fancyhead[L]{\textsf{The 2024 ICPC China Shaanxi National Invitational Programming Contest}}
\fancyhead[R]{\textsf{2024/05/19}}
\renewcommand{\headrulewidth}{1pt}
\renewcommand{\footrulewidth}{1pt}

\setlength{\parindent}{0pt}
\usepackage{setspace}
\setstretch{1.25} % 自定义行距


\begin{document}
\begin{titlepage}
    \centering\huge\textbf{The 2024 ICPC China\\ Shaanxi National Invitational \\Programming Contest}
    \vspace{0.03\textheight}
    
    \centering{Contest Session}

    \vspace{0.05\textheight}
    
    \centering\Large{May 19, 2024}

    \begin{figure}[h]
        \centering
        \includegraphics[width=0.3\linewidth]{西北工业大学-logo-1024px.png}
    \end{figure}

    \vspace{0.05\textheight}

    \centering{Problem List}

    
    \begin{table}[h]
        \centering
        \begin{tabular}{|c|c|}
        \hline
             A&Guess The Tree \\
             B&Turn Off The Lights \\
             C&Fix the Tree \\
             D&Make Them Straight \\
             E&Dumb Robot \\
             F&XOR Game \\
             G&The Last Cumulonimbus Cloud \\
             H&Holes and Balls \\
             I&Smart Quality Inspector \\
             J&Triangle \\
             K&Yet Another Maximum Matching Counting Problem \\
             L&Rubbish Sorting \\
             M&Chained Lights \\
        \hline
        \end{tabular}
    \end{table}
    \small{This problem set should contain 13 (thirteen) problems on 13 (thirteen) numbered pages.\\Please inform a runner immediately if something is missing from your problem set.}
\end{titlepage}

\textbf{\Large\textsf{Problem A. Guess The Tree}}
\vspace{0.01\textheight}

There is a full binary tree with $n$ levels(so it has exactly $2^n-1$ nodes). Each node has an integer ID from $1$ to $2^n-1$, and the $2^n-1$ IDs form an arrangement from $1$ to $2^n-1$, but you don't know.

You need to find the IDs of the $2^n-1$ nodes using at most $4800$ queries.

\vspace{0.01\textheight}
\textbf{\textsf{Input}}
\vspace{0.01\textheight}

The first line contains one integer $n(1\leq n\leq 10)$, the levels of the full binary tree.

To ask a query, you need to pick two nodes with IDs $u,v(1\leq u,v\leq 2^n-1)$, and print the line of the following form:

\texttt{> "? u v"}

After that, you will receive:

\texttt{> "t"}

The lowest common ancestor's ID of $u$ and $v$.

You can ask at most $4800$ queries.

If you have found the structure of the tree, print a line of the following form:

\texttt{"! $f_1\ f_2\ f_3\ f_4$ ... $f_{2^n-1}$"}

$f_i$ means the i-th node's father's ID. If it has no father, then $f_i=-1$.

After printing a query or the answer for a test case, do not forget to output the end of line and flush the output. Otherwise, you will get the verdict 'Idleness Limit Exceeded'. To do this, use:

\texttt{fflush(stdout)} or \texttt{cout.flush()} in C++;

\texttt{System.out.flush()} in Java;

\texttt{stdout.flush()} in Python.

The interactor in this task is not adaptive.

\vspace{0.01\textheight}
\textbf{\textsf{Output}}
\vspace{0.01\textheight}


None

\vspace{0.01\textheight}
\textbf{\textsf{Example}}
\vspace{0.01\textheight}

\begin{table}[h]
    \centering\texttt{
        \begin{tabularx}{\textwidth}{|X|X|}
            \hline
             standard input&standard output\\
             \hline
             2&\\
             &? 1 2\\
             3&\\
             &? 2 3\\
             3&\\
             &? 1 3\\
             3&\\
             &! 3 3 -1\\
            \hline
        \end{tabularx}
    }
\end{table}

\vspace{0.01\textheight}
\textbf{\textsf{Note}}
\vspace{0.01\textheight}

In this case, the tree's root is $3$, it's two sons are $1$ and $2$.

For any query \texttt{"? a b"},if $a\neq b$, the jury will return answer $3$.

So we found the tree's root is $3$ .

\newpage

%-------------------------------------------

\textbf{\Large\textsf{Problem B. Turn Off The Lights}}
\vspace{0.01\textheight}

Kitty has $n^2$ lights, which form an $n\times n$ matrix.

One day, Kitty found that some of these lights were on, and some were off. Kitty wants to turn them all off.

To achieve her goal, Kitty can perform three types of operations:

- (1) Choose a row, reverse the state of this row. It means if a light of this row is on, after this operation, it is now off. If a light of this row is off, after this operation, it is now on.

- (2) Choose a column, reverse the state of this column. It means if a light of this column is on, after this operation, it is now off. If a light of this column is off, after this operation, it is now on.

- (3) Choose exactly one light, reverse the state of this light. \textbf{This operation can only be performed not more than $k$ times.}

For the current state, help Kitty achieve her goal within $3n$ operations.

\vspace{0.01\textheight}
\textbf{\textsf{Input}}
\vspace{0.01\textheight}

The first line contains two integers $n(1\leq n\leq 1000),k(0\leq k < n)$, indicating as described above.

Then $n$ lines follow, each line has exactly $n$ numbers, $0$ represents that the light is turned off at this time, while $1$ represents the opposite.

The $y$-th number of the $(x+1)$-th line in input means the light at coordinate $(x,y)$.

\vspace{0.01\textheight}
\textbf{\textsf{Output}}
\vspace{0.01\textheight}

If Kitty can not achieve her goal,print $-1$ in a single line.

Otherwise, print $M(0\leq M\leq 3n)$ in the first line, indicating the number of operations she needs to perform.

The next $M$ lines, each line contains $2$ integers $x,y$, separated by white space.

If $1\leq x\leq n,1\leq y\leq n$, it means Kitty will reverse the light at coordinate $(x,y)$.

If $x=0,1\leq y\leq n$, it means Kitty will reverse all lights at coordinates $(z,y)1\leq z\leq n$.

If $1\leq x\leq n,y=0$, it means Kitty will reverse all lights at coordinates $(x,z)1\leq z\leq n$.

If there are multiple answers, print any of them.

\vspace{0.01\textheight}
\textbf{\textsf{Example}}
\vspace{0.01\textheight}

\begin{table}[h]
    \centering\texttt{
        \begin{tabularx}{\textwidth}{|X|X|}
            \hline
             standard input&standard output\\
             \hline
             2 0&2\\
             0 1&0 2\\
             1 0&2 0\\
            \hline
             3 1&-1\\
             1 0 0&\\
             0 1 0&\\
             0 0 1&\\
            \hline
        \end{tabularx}
    }
\end{table}

\newpage

%----------------------------

\textbf{\Large\textsf{Problem C. Fix the Tree}}
\vspace{0.01\textheight}

You are given a tree consisting of $n$ vertices. Each vertex $i$ of this tree has a value $w_i$ assigned to it.

Now the vertex $u$ will be broken. Once it's broken, vertex $u$ and all edges with one end at vertex $u$ will be removed from the tree.

To make the tree connected, you can do the following operation any number of times(possibly zero) in any order:

- First choose two vertices $u$ and $v$ from the tree;

- Then pay $w_u+w_v$ coins, and add an edge between vertices $u$ and $v$;

- At last, replace $w_u+1$ with $w_u$, replace $w_v+1$ with $w_v$.

Your task is to calculate the minimum number of coins to be paid.

But you don't know which vertex $u$ is, so you need to find the answer for each $1\le u\le n$. Please answer all the queries independently.

\vspace{0.01\textheight}
\textbf{\textsf{Input}}
\vspace{0.01\textheight}

The first line contains a single integer $n(2\le n\le 10^6)$ --- the number of vertices in this tree.

The next line contains $n$ numbers, the $i$ -th number is $w_i(1\le w_i\le n)$.

The next $n-1$ lines contain a description of the tree's edges. The $i$ -th of these lines contains two integers $u_i$ and $v_i(1\le u_i,v_i\le n) $ --- the numbers of vertices connected by the $i$ -th edge.

It is guaranteed that the given edges form a tree.

\vspace{0.01\textheight}
\textbf{\textsf{Output}}
\vspace{0.01\textheight}

Print $n$ integers, the $i$ -th integer is the answer when $u=i$.

\vspace{0.01\textheight}
\textbf{\textsf{Example}}
\vspace{0.01\textheight}

\begin{table}[h]
    \centering\texttt{
        \begin{tabularx}{\textwidth}{|X|X|}
            \hline
             standard input&standard output\\
             \hline
             6&4 4 0 0 0 0\\
             1 1 1 1 1 1&\\
             1 2&\\
             1 3&\\
             1 4&\\
             2 5&\\
             2 6&\\
            \hline
             4&12 0 0 0\\
             1 2 3 4&\\
             1 2&\\
             1 3&\\
             1 4&\\
            \hline
             7&5 12 16 0 0 0 0\\
             1 2 3 4 5 6 7&\\
             1 2&\\
             1 3&\\
             2 4&\\
             2 5&\\
             3 6&\\
             3 7&\\
            \hline
        \end{tabularx}
    }
\end{table}

\newpage

%-------------------------------------------

\textbf{\Large\textsf{Problem D. Make Them Straight}}
\vspace{0.01\textheight}

There is a sequence $a$ of non-negative integers of length $n$, the $i$-th element of it is $a_i(1\leq i\leq n)$.

A sequence is defined as 'good' if and only if there exists a non negative integer $k(0\leq k\leq 10^6)$ that satisfies the following condition:

$a_{i}=a_{1}+(i-1)k$ for all $1\leq i\leq n$.

To make this sequence 'good', for each $i(1\leq i\leq n)$, you can do nothing, or pay $b_i$ coin to replace $a_i$ with any non-negative integer.

The question is, what is the minimum cost to make this sequence 'good'.

\vspace{0.01\textheight}
\textbf{\textsf{Input}}
\vspace{0.01\textheight}

The first line contains an integer $n(1\leq n\leq 2\times 10^5)$, described in the statement.

The second line contains $n$ integers $a_1,...,a_n(0\leq a_i\leq 10^6)$.

The third line contains $n$ integers $b_1,...,b_n(0\leq b_i\leq 10^6)$.

\vspace{0.01\textheight}
\textbf{\textsf{Output}}
\vspace{0.01\textheight}

One integer, the answer.

\vspace{0.01\textheight}
\textbf{\textsf{Example}}
\vspace{0.01\textheight}

\begin{table}[h]
    \centering\texttt{
        \begin{tabularx}{\textwidth}{|X|X|}
            \hline
             standard input&standard output\\
             \hline
             5&2\\
             1 4 3 2 5&\\
             1 1 1 1 1&\\
             \hline
             5&3\\
             1 4 3 2 5&\\
             1 9 1 1 1&\\
             \hline
        \end{tabularx}
    }
\end{table}

\newpage

%-------------------------------------------

\textbf{\Large\textsf{Problem  E. Dumb Robot}}
\vspace{0.01\textheight}

You have a dumb robot, and you are going to let it play games with $n$ robots.

There is a matrix $A$ with three rows and three columns in the game. We call the number of row $i$ and column $j$ of this matrix $A_{i,j}$. The game goes like this:

Two players each choose an integer from $[1,3]$ at the same time. We call the number your robot chooses $i$, and the number the other robot chooses $j$.
The score is $A_{i,j}$.
In game $i$, your robot will play this game with the $i$ -th robot. The $i$ -th robot has a probability of choosing $1$ as $p_{i,1}$, a probability of choosing $2$ as $p_{i,2}$, and a probability of choosing $3$ as $p_{i,3}$.

Your goal is to make the expected value of the score not negative in each game. But your robot is very dumb, so it will choose $1$ with probability $q_1$, $2$ with probability $q_2$, and $3$ with probability $q_3$, and you don't know the value of $q_1,q_2,q_3$.

We all know that $q_1+q_2+q_3=1$. If $q_1,q_2,q_3$ are chosen uniformly at random from a set of all possible cases, please calculate the probability that you will reach your goal.

\vspace{0.01\textheight}
\textbf{\textsf{Input}}
\vspace{0.01\textheight}

The first line contains one integer $n$($1\le n\le10^4$).

Each of the next $3$ lines contains $3$ integers, the $j$ -th integer in the $i$ -th of these lines is $A_{i,j}$($-20\le A_{i,j}\le20$).

Each of the next $n$ lines contains $3$ real numbers, the $j$ -th number in the $i$ -th of these lines is $p_{i,j}$. It is guaranteed that $p_{i,1}+p_{i,2}+p_{i,3}=1$ and $0\le p_{i,j}$.

\vspace{0.01\textheight}
\textbf{\textsf{Output}}
\vspace{0.01\textheight}

Output the answer to the problem. It is guaranteed that the answer will never be $0$.

Your answer is considered correct if its absolute or relative error does not exceed $10^{-2}$. Formally, let your answer be $a$, and the jury's answer be $b$. Your answer is accepted if and only if $\frac{|a-b|}{max(1,|b|)} \leq 10^{-2}$.

\vspace{0.01\textheight}
\textbf{\textsf{Example}}
\vspace{0.01\textheight}

\begin{table}[h]
    \centering\texttt{
        \begin{tabularx}{\textwidth}{|X|X|}
            \hline
             standard input&standard output\\
             \hline
             1&0.748252\\
             1 1 1&\\
             -1 2 1&\\
             0 -3 2&\\
             0.1 0.6 0.3&\\
            \hline
             8&0.111111\\
             1 3 -2&\\
             0 0 2&\\
             -2 2 1&\\
             0.1 0.3 0.6&\\
             0 0 1&\\
             0.5 0.2 0.3&\\
             0 0 1&\\
             1 0 0&\\
             0 0 1&\\
             0.33 0.33 0.34&\\
             0.16 0.16 0.68&\\
            \hline
        \end{tabularx}
    }
\end{table}

\vspace{0.01\textheight}
\textbf{\textsf{Note}}
\vspace{0.01\textheight}

In example $1$, for example, $(q_1=1,q_2=0,q_3=0)$ is ok. In this case, Your robot will always choose $1$, so no matter what number will robot $1$ choose, the score will always be $1$, which is enough to reach your goal.

\newpage

%-------------------------------------------

\textbf{\Large\textsf{Problem F. XOR Game}}
\vspace{0.01\textheight}

Alice and Bob are playing a game against each other.

In front of them are a multiset $\{a_i\}$ of non-negative integers and a single integer $x$. Each number in $a$ is $0$ or $2^i(0\le i<k)$ before the game.

This game will be a turn-based game, and Alice will go first. In one person's turn, he or she will choose an integer from $a$. Let this number be $p$. Then this person will choose whether or not to make $x\gets x\oplus p$, then remove $p$ from $a$. Here, operation $\oplus$ means bitwise xor.

Alice wants to make $x$ as big as possible, and Bob wants to make $x$ as small as possible.

You are a bystander who wants to know the final value of $x$. However, the size of $a$ is a huge number. Formally, there are $b_i$ numbers whose values are $2^i$ in $a$ for all $0\le i<k$, and $z$ numbers whose values are $0$. But you still want to challenge this impossible problem.

If Alice and Bob are smart enough, please output the final value of $x$.

\vspace{0.01\textheight}
\textbf{\textsf{Statement updated}}
\vspace{0.01\textheight}

$z$ is the number of numbers whose values are $0$.

\vspace{0.01\textheight}
\textbf{\textsf{Input}}
\vspace{0.01\textheight}

The first line contains two integers $k,z(1\le k\le10^5,0\le z\le 10^9)$.

The next line contains $k$ integers, the $i$ -th integer is $b_{i-1}(0\le b_{i-1}\le10^9)$.

\vspace{0.01\textheight}
\textbf{\textsf{Output}}
\vspace{0.01\textheight}

Output the answer in binary format. Note that you should output exactly $k$ digit from high to low even though this number has leading $0$s.

\vspace{0.01\textheight}
\textbf{\textsf{Example}}
\vspace{0.01\textheight}

\begin{table}[h]
    \centering\texttt{
        \begin{tabularx}{\textwidth}{|X|X|}
            \hline
             standard input&standard output\\
             \hline
             1 0&1\\
             3&\\
             \hline
             2 0&11\\
             2 1&\\
             \hline
             2 0&00\\
             2 2&\\
            \hline
        \end{tabularx}
    }
\end{table}

\newpage

%-------------------------------------------

\textbf{\Large\textsf{Problem G. The Last Cumulonimbus Cloud}}
\vspace{0.01\textheight}

Every April, the city is always shrouded under cumulonimbus clouds.

This city is connected by $n$ buildings and $m$ two-way streets. In order to facilitate people's travel, any two buildings can directly or indirectly reach each other through the streets. At the same time, no street connects the same building, and there is at most one street that connects each pair of buildings.

The pace of life in this city is very slow because the city layout is not very bulky.

Specifically,if we consider this city as an undirected graph $G$ ,it is guaranteed that for any non empty subgraph in this graph,there is at least one building inside it that connects up to 10 streets within the subgraph.

The rain is not stopping, and the number of cumulonimbus clouds is constantly increasing. At the beginning, there are $a_i$ cumulonimbus clouds above the $i$ -th building, but in the following $q$ days, one of the following two events will occur every day:

- $\text{1 x v}$ $v$ cumulonimbus clouds have been added over the $x$ -th building.

- $\text{2 x}$ you need to report how many cumulonimbus clouds are in total over all buildings directly connected to building $x$.

\vspace{0.01\textheight}
\textbf{\textsf{Input}}
\vspace{0.01\textheight}

The first line contains three integers $n,m,q(1\le n\le 3\times 10^5,1\leq m\leq 3\times 10^6, 1\leq q\leq 2\times 10^6)$.

Each of the next $m$ lines contains two integers $x,y(1\leq x,y\leq n,x\neq y)$, which represents a street connecting the $x$ -th and $y$ -th buildings.

Each of the next $n$ lines contains an integer $a_i(0\leq a_i\leq 100)$.

Each of the next $q$ lines contains two or three integers, if the first integer is $1$, it represents a first type of event, and the next two integers represent $x,v(0\leq v\leq 100)$. If the first integer is $2$, it represents a second type of event, the next integer represents $x$.

\vspace{0.01\textheight}
\textbf{\textsf{Output}}
\vspace{0.01\textheight}

Several rows, each representing a query result for a second type of event.

\vspace{0.01\textheight}
\textbf{\textsf{Example}}
\vspace{0.01\textheight}

\begin{table}[h]
    \centering\texttt{
        \begin{tabularx}{\textwidth}{|X|X|}
            \hline
             standard input&standard output\\
             \hline
             4 6 10&8\\
            2 4&7\\
            2 3&17\\
            4 3&20\\
            3 1&19\\
            4 1&26\\
            2 1&25\\
            0&\\
            7&\\
            1&\\
            6&\\
            2 4&\\
            2 2&\\
            1 3 3&\\
            2 1&\\
            1 1 9&\\
            2 4&\\
            2 2&\\
            1 3 6&\\
            2 4&\\
            2 2&\\
            \hline
        \end{tabularx}
    }
\end{table}

\newpage

%-------------------------------------------

\textbf{\Large\textsf{Problem H. Holes and Balls}}
\vspace{0.01\textheight}

You are given $n$ balls, the $i$ -th ball's value is $p_i$. It's guaranteed that $p_1,p_2,\dots,p_n$ is a permutation of $1,2,3\dots,n$.
    
    
    
There is also a rooted tree of $n$ vertices, each of the vertices is a hole, and each hole can only hold one ball.
    
    
    
The tree's root is the first vertex.
    
    
    
Now you need to fill the holes with the balls.
    
You need to throw each ball in order of $1$ to $n$ in the following steps:

- 1. Throw the ball into vertex $1$.

- 2. Let the vertex where the ball is be $p$.

- 3. If the $p$ -th vertex has already been filled with other balls, you need to choose a vertex $x$ and throw the ball into the $x$ -th vertex, then return to step $2$. You need to guarantee that the $x$ -th vertex is the $p$ -th vertex's son and at least one vertex in the subtree of the $x$ -th vertex is not filled.

- 4. Otherwise, the ball will fill the $p$ -th vertex.

After throwing all the balls, let $a_i$ express the value of the ball in the $i$ -th vertex.
    
    
    
You need to find the minimum lexicographical order of $\{a_n\}$.
    
    
    
We define $dep_i$ as the number of vertices on the path from the $i$ -th vertex to the tree's root(the first vertex).
    
    
    
Specially, for any two vertices $x<y$, it's guaranteed that $dep_x\le dep_y$.

\vspace{0.01\textheight}
\textbf{\textsf{Input}}
\vspace{0.01\textheight}

The first line contains a single integer $n(1\le n\le 5\times 10^5)$ - the number of vertices in this tree.
    
    
    
The next line contains $n$ numbers, the $i$ -th number is $p_i(1\le p_i\le n)$. It's guaranteed that $p_1,p_2,\dots,p_n$ is a permutation of $1,2,3\dots,n$.
    
    
    
The next $n-1$ lines contain a description of the tree's edges. The $i$ -th of these lines contains two integers $u_i$ and $v_i(1\le u_i,v_i\le n) $ - the numbers of vertices connected by the $i$ -th edge.
    
    
    
It is guaranteed that the given edges form a tree.
    
    
    
And for any vertices $x<y$, it's guaranteed that $dep_x\le dep_y$.

\vspace{0.01\textheight}
\textbf{\textsf{Output}}
\vspace{0.01\textheight}

Output $n$ integers, the minimum lexicographical order of $\{a_n\}$.

\vspace{0.01\textheight}
\textbf{\textsf{Example}}
\vspace{0.01\textheight}

\begin{table}[h]
    \centering\texttt{
        \begin{tabularx}{\textwidth}{|X|X|}
            \hline
             standard input&standard output\\
             \hline
5&3 1 5 4 2\\
3 1 5 4 2&\\
1 2&\\
2 3&\\
3 4&\\
4 5&\\
            \hline
9&9 2 1 3 6 4 8 5 7\\
9 2 6 3 5 7 1 4 8&\\
1 2&\\
1 3&\\
2 4&\\
2 5&\\
3 6&\\
3 7&\\
4 8&\\
4 9&\\
            \hline
        \end{tabularx}
    }
\end{table}

\newpage

%-------------------------------------------

\textbf{\Large\textsf{Problem I. Smart Quality Inspector}}
\vspace{0.01\textheight}

Ella has a factory. One day, her factory is facing a product quality inspection.
    
    
    
Her factory has $N$ production lines. Among the $N$ production lines, $N-K$ of them are qualified, and the other $K$ lines are unqualified. The fine of the $i$ -th$(1\leq i\leq K)$ unqualified line is $i$ Yuan.
    
    
    
There are $M$ quality inspectors here. For the $j$ -th$(1\leq j\leq M)$ quality inspector, he will inspect from the $l_i$ -th line to the $r_i$ -th line and find the unqualified production line with the largest fine among them, then impose this fine on Ella.
    
    
    
Ella does not want to receive so many fines, so she decides to renumber the $N$ production lines to receive the least amount of fines. Please help her.
    
    
    
In simple terms:
    
    
    
You have a sequence $A$ of length $N$, $A=[1,2,3,...,K,0,0,0,...,0]$. Here $N,K$ are given.
    
    
    
There are $M$ pairs of integers, each pair consists of two numbers $l_i,r_i$.
    
    
    
You need to rearrange sequence $A$ to minimize the following:
    
    
    
$$\sum_{i=1}^M \max_{j=l_i}^{r_i} (A_{j})$$

\vspace{0.01\textheight}
\textbf{\textsf{Input}}
\vspace{0.01\textheight}

The first line contains three integers $N,K,M(1\leq K\leq N\leq 20,1\leq M\leq 10^5)$, described in the statement.
    
    
    
Then $M$ lines, the $i$ -th line of them contains two integers $l_i,r_i(1\leq l_i\leq r_i\leq N)$.

\vspace{0.01\textheight}
\textbf{\textsf{Output}}
\vspace{0.01\textheight}

An integer indicates the answer.

\vspace{0.01\textheight}
\textbf{\textsf{Example}}
\vspace{0.01\textheight}

\begin{table}[h]
    \centering\texttt{
        \begin{tabularx}{\textwidth}{|X|X|}
            \hline
             standard input&standard output\\
             \hline
4 4 3&10\\
1 2&\\
3 4&\\
1 4&\\
            \hline
        \end{tabularx}
    }
\end{table}

\newpage

%-------------------------------------------

\textbf{\Large\textsf{Problem J. Triangle}}
\vspace{0.01\textheight}

There are three points $A(a, 0)$, $B(0, b)$, $C(0, 0)$ in the plane rectangular coordinate system. Define the size of triangle $ABC$ as the number of squares that belong to it.
    
    
    
For integers $x,y$, one square is defined by four points $(x, y)$, $(x + 1, y)$, $(x, y + 1)$, $(x + 1, y + 1)$. We consider a square to belong to a triangle if half or more of it is inside the triangle.
    
    
    
For example, when $a = 8, b = 6$, the size of triangle $ABC$ will be $24$. The image below shows which squares are counted in the triangle.

\begin{figure}[h]
    \centering
    \includegraphics[width=0.5\linewidth]{J.png}
\end{figure}

\vspace{0.01\textheight}
\textbf{\textsf{Input}}
\vspace{0.01\textheight}

The first and only line contains two integers $a\ (1\le a\le 10^6)$ and $b\ (1\le b\le 10^6)$.

\vspace{0.01\textheight}
\textbf{\textsf{Output}}
\vspace{0.01\textheight}

Print an integer: the size of triangle $ABC$.

\vspace{0.01\textheight}
\textbf{\textsf{Example}}
\vspace{0.01\textheight}

\begin{table}[h]
    \centering\texttt{
        \begin{tabularx}{\textwidth}{|X|X|}
            \hline
             standard input&standard output\\
             \hline
             6 8&24\\
            \hline
             5 5&15\\
            \hline
             1 999999&500000\\
            \hline
        \end{tabularx}
    }
\end{table}

\newpage

%-------------------------------------------

\textbf{\Large\textsf{Problem K. Yet Another Maximum Matching Counting Problem}}
\vspace{0.01\textheight}

There is a two-dimensional plane.
    
    
    
You have a set of points $\{(x_i,y_i)\}$ that satisfies $1\le x_i\le n, 1\le y_i\le m$ (Both $x_i$ and $y_i$ are integers), and there are no two points with the same coordinates.
    
    
    
If two points have the same horizontal or vertical coordinates, we will connect an edge between these two points. This forms a graph.
    
    
    
You need to find the sum of the maximum number of matches in the graphs formed by all possible $2^{nm}-1$ non empty sets, and output the result modulo $998244353$.
    
    
    
Here, the maximum number of matches in a graph is defined as: selecting the most edges so that there are no common vertices between any two edges.

\vspace{0.01\textheight}
\textbf{\textsf{Input}}
\vspace{0.01\textheight}

There are multiple testcases in this problem.
    
    
    
The first line contains an integer $T(1\le T\le 100)$, which represents the number of testcases.
    
    
    
Each of the testcases contains two integers $n,m(1\leq n,m\leq 500)$.

\vspace{0.01\textheight}
\textbf{\textsf{Output}}
\vspace{0.01\textheight}

For each of the testcases, print an integer representing the result modulo $998244353$.

\vspace{0.01\textheight}
\textbf{\textsf{Example}}
\vspace{0.01\textheight}

\begin{table}[h]
    \centering\texttt{
        \begin{tabularx}{\textwidth}{|X|X|}
            \hline
             standard input&standard output\\
             \hline
10&0\\
1 1&1\\
1 2&10\\
2 2&241456\\
4 4&964\\
3 3&200419152\\
5 5&448\\
1 8&985051144\\
20 20&370696900\\
100 100&357517517\\
500 500&\\
            \hline
        \end{tabularx}
    }
\end{table}

\newpage

%-------------------------------------------

\textbf{\Large\textsf{Problem L. Rubbish Sorting}}
\vspace{0.01\textheight}

Bob has many pieces of rubbish. One day, he wants to sort them.
    
    
    
For every piece of rubbish, its type is expressed as a positive integer.
    
    
    
He has $q$ operations. For each operation, it is one of the following two operations.

- \texttt{1 s x} He tells you that the piece of rubbish named $s$ has a type of $x$.

- \texttt{2 s} He wants to ask you the type of rubbish $s$.


    
But his memories are not always accurate.
    
    
    
For each operation $2$, $s$ may not have appeared in the previous operation $1$s.
    
    
    
We define the similarity of two strings $s_1$ and $s_2$ as $\sum_{i=1}^{\min\{|s_1|,|s_2|\}} [s_{1,i}=s_{2,i}]$.
    
    
    
Here all the strings' indexes start at $1$.
    
    
    
For a string $s$, its type is the type of string which has the maximum similarity to $s$ among all the strings that have appeared in the previous operations $1$s. Note that if there are multiple strings that all have the maximum similarity to $s$, the type of $s$ is the minimum type of these strings' type.
    
    
    
Now, he wants you to solve this problem.

\vspace{0.01\textheight}
\textbf{\textsf{Input}}
\vspace{0.01\textheight}

The first line contains an integer $q(1\le q\le 3\times 10^5)$, which is the number of operations.
    
    
    
Next $q$ lines contain operations, one per line. They correspond to the description given in the statement.
    
    
    
It is guaranteed that for every operation $2$ there is at least one operation $1$ before it.
    
    
    
But some pieces of rubbish will have more than one type, you can consider it as the minimum type you have read.
    
    
    
The rubbish's names only consist of lowercase Latin letters.
    
    
    
$1 \le |s| \le 5, 1 \le x \le 10^9$

\vspace{0.01\textheight}
\textbf{\textsf{Output}}
\vspace{0.01\textheight}

For every operation $2$, you should print an integer in a single line that is the rubbish $s$'s type.

\vspace{0.01\textheight}
\textbf{\textsf{Example}}
\vspace{0.01\textheight}

\begin{table}[h]
    \centering\texttt{
        \begin{tabularx}{\textwidth}{|X|X|}
            \hline
             standard input&standard output\\
             \hline
4&1\\
1 aaa 1&2\\
2 aa&\\
1 ab 2&\\
2 bb&\\
            \hline
        \end{tabularx}
    }
\end{table}

\newpage

%-------------------------------------------

\textbf{\Large\textsf{Problem M. Chained Lights}}
\vspace{0.01\textheight}

You have $n$ lights in a row. Initially, they are all off.
    
    
    
You are going to press these $n$ lights one by one. When you press light $i$, light $i$ will switch its state, which means it will turn on if it's off and turn off if it's on, and then for every $j$ satisfied $i|j,i< j\le n$, press light $j$ once.
    
    
    
For example, if $n=4$, when you press light $1$, light $1$ will turn on, and then you will press light $2,3,4$. Since you pressed light $2$, light $2$ will turn on and you will press light $4$, which will cause light 4 to turn on. After all the operations, lights $1,2,3$ will be turned on and light $4$ is still off.
    
    
    
You will press these $n$ lights and do the operations as mentioned above one by one. After all the operations, you want to know whether light $k$ is on or off.
    
    
    
You can also use the following code to understand the meaning of the problem:

\begin{lstlisting}[language=C++, basicstyle=\ttfamily]
void press(int x)
{
    light[x]^=1;
    for (int y=x+x;y<=n;y+=x) press(y);
}
for (int i=1;i<=n;i++) press(i);
\end{lstlisting}


\vspace{0.01\textheight}
\textbf{\textsf{Input}}
\vspace{0.01\textheight}

There are multiple testcases.
    
    
    
The first line contains an integer $T(1\le T\le10^5)$, which represents the number of testcases.
    
    
    
Each of the testcases contains two integers $n,k(1\le k\le n\le10^6)$ in a single line.

\vspace{0.01\textheight}
\textbf{\textsf{Output}}
\vspace{0.01\textheight}

For each testcase, if light $k$ is turned on in the end, output $\text{YES}$, otherwise output $\text{NO}$.

\vspace{0.01\textheight}
\textbf{\textsf{Example}}
\vspace{0.01\textheight}

\begin{table}[h]
    \centering\texttt{
        \begin{tabularx}{\textwidth}{|X|X|}
            \hline
             standard input&standard output\\
             \hline
2&YES\\
1 1&NO\\
3 2&\\
            \hline
        \end{tabularx}
    }
\end{table}

\newpage

%-------------------------------------------

\end{document}
